\chapter{Data-at-rest encryption}
\url{https://wiki.archlinux.org/title/Data-at-rest_encryption#Block_device_encryption}
\section{Introduction}

\subsection{Stacked filesystem encryption}
Stacked filesystem encryption solutions are implemented as a layer that stacks
on top of an existing filesystem, causing all files written to an
encryption-enabled folder to be encrypted on-the-fly before the underlying
filesystem writes them to disk, and decrypted whenever the filesystem reads
them from disk. This way, the files are stored in the host filesystem in
encrypted form (meaning that their contents, and usually also their file/folder
names, are replaced by random-looking data of roughly the same length), but
other than that they still exist in that filesystem as they would without
encryption, as normal files / symlinks / hardlinks / ldots

The way it is implemented, is that to unlock the folder storing the raw
encrypted files in the host filesystem ("lower directory"), it is mounted
(using a special stacked pseudo-filesystem) onto itself or optionally a
different location ("upper directory"), where the same files then appear in
readable form - until it is unmounted again, or the system is turned off.

Available solutions in this category are
\href{https://wiki.archlinux.org/title/ECryptfs}{eCryptfs} and
\href{https://wiki.archlinux.org/title/EncFS}{EncFS}.

\subsection{Block device encryption}

Block device encryption methods, on the other hand, operate below the
filesystem layer and make sure that everything written to a certain block
device (i.e. a whole disk, or a partition, or a file acting as a loop device)
is encrypted. This means that while the block device is offline, its whole
content looks like a large blob of random data, with no way of determining what
kind of filesystem and data it contains. Accessing the data happens, again, by
mounting the protected container (in this case the block device) to an
arbitrary location in a special way.

\begin{itemize}
    \item Loop-AES:  is a descendant of cryptoloop and is a secure and fast
            solution to system encryption. However, loop-AES is considered less
            user-friendly than other options as it requires non-standard kernel
            support.

    \item dm-crypt:  is the standard device-mapper encryption functionality
        provided by the Linux kernel. It can be used directly by those who like
        to have full control over all aspects of partition and key management.
        The management of dm-crypt is done with the \verb+cryptsetup+ userspace
        utility. It can be used for the following types of block-device
        encryption: LUKS (default), plain, and has limited features for loopAES
        and Truecrypt devices: 
        \begin{itemize}
            \item LUKS, used by default, is an additional convenience layer
                which stores all of the needed setup information for dm-crypt
                on the disk itself and abstracts partition and key management
                in an attempt to improve ease of use and cryptographic
                security.i A partition using LUKS consists of two parts: the
                header and the encrypted data. The header contains keys without
                which it is practically impossible to recover the data.
            \item plain dm-crypt mode, being the original kernel functionality,
                does not employ the convenience layer. It is more difficult to
                apply the same cryptographic strength with it. When doing so,
                longer keys (passphrases or keyfiles) are the result. It has,
                however, other advantage. 
        \end{itemize}
    \item  TrueCrypt/VeraCrypt: A portable format, supporting encryption of
        whole disks/partitions or file containers, with compatibility across
        all major operating systems.
\end{itemize}


\section{dm-crypt}

\subsection{Encrypting a non-root file system}
To encrypt a device that is not used for booting a system, like a partition or a file container.

Encrypting a secondary filesystem usually protects only sensitive data, while
leaving the operating system and program files unencrypted. This is useful for
encrypting an external medium, such as a USB drive, so that it can be moved to
different computers securely. One might also choose to encrypt sets of data
separately according to who has access to it.

\subsubsection{Drive preparation}

Before encrypting a drive, it is recommended to perform a secure erase of the
disk by overwriting the entire drive with random data. To prevent cryptographic
attacks or unwanted file recovery, this data is ideally indistinguishable from
data later written by dm-crypt.

dm-crypt wipe on an empty disk or partition:
\begin{verbatim}
# create a temporary encrypted container on 
# the partition (sdXY) or complete device (sdX)
cryptsetup open --type plain -d /dev/urandom /dev/<block-device> to_be_wiped
# verify that it exists
lsblk
# Wipe the container with zero
dd if=/dev/zero of=/dev/mapper/to_be_wiped status=progress
# close the temporary container
cryptsetup close to_be_wiped
\end{verbatim}

\subsubsection{Parititonning}

Create the partition which will contain the encrypted container then:
\begin{verbatim}
#  setup the LUKS header with: 
# cryptsetup options luksFormat device
sudo /usr/bin/cryptsetup -y -v luksFormat /dev/sdc1

# gain access to the encrypted partition
# cryptsetup open <device> <name>
sudo /usr/bin/cryptsetup luksOpen /dev/sdc1 mykey

# create a file system 
# mkfs.fstype /dev/mapper/name
sudo mkfs.ext4 /dev/mapper/mykey
\end{verbatim}

\subsubsection{Manual mounting and unmounting}
\begin{verbatim}
cryptsetup open <device> <name>
mount -t fstype /dev/mapper/<name> <mount_point>

sudo /usr/bin/cryptsetup luksOpen /dev/sdc1 mykey
sudo mount -t ext4 /dev/mapper/mykey /tmp
\end{verbatim}

\begin{verbatim}
sudo umount /tmp
sudo cryptsetup close mykey
\end{verbatim}

\subsubsection{Automated unlocking and mounting}
iThere are three different solutions for automating the process of unlocking
the partition and mounting its filesystem.

{\bf At boot time} : Using the \verb+/etc/crypttab+ configuration file,
unlocking happens at boot time by systemd's automatic parsing. This is the
recommended solution to use one common partition for all user's automatically
mount another encrypted block device.

{\bf On user login} Using \verb+pam-exec+ (recommended to have a single user's
home directory on a partition) or using \verb+pam_mount+.

\subsubsection{Passphrase rotation}
Passphrase can be changed within one step with the \verb+luksChangeKey+. But in
order to prevent typo it can be done in two steps:

First add a new passphrase to the volume.
\begin{verbatim}
cryptsetup luksAddKey /dev/<device>
\end{verbatim}

Then remove previous passphrase.
\begin{verbatim}
cryptsetup luksRemoveKey /dev/<device>
\end{verbatim}


\subsection{Encrypting an entire system}
To encrypt an entire system, in particular a root partition. Several scenarios
are covered, including the use of dm-crypt with the LUKS extension, plain mode
encryption and encryption and LVM.

