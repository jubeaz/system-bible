\chapter{Tails OS}

\url{https://tails.boum.org}

\section{tails in a VM}
\subsection{Security considerations}

Running Tails inside a virtual machine has various security implications.
Depending on the host operating system and your security needs, running Tails
in a virtual machine might be dangerous.

    Both the host operating system and the virtualization software are able to
    monitor what you are doing in Tails.

    If the host operating system is compromised with a software keylogger or
    other malware, then it can break the security features of Tails.

    Only run Tails in a virtual machine if both the host operating system and
    the virtualization software are trustworthy.

    Traces of your Tails session are likely to be left on the local hard disk.
    For example, host operating systems usually use swapping (or paging) which
    copies part of the RAM to the hard disk.

    Only run Tails in a virtual machine if leaving traces on the hard disk is
    not a concern for you.

The Tails virtual machine does not modify the behaviour of the host operating
system and the network traffic of the host is not anonymized. The MAC address
of the computer is not modified by the MAC address anonymization feature of
Tails when run in a virtual machine. 
\subsection{libvirt}
\begin{verbatim}
To create a new virtual machine, choose File ▸ New Virtual Machine.
In step 1, choose Local install media (ISO image or CDROM).
In step 2:
    Choose ISO image, then Browse..., and Browse Local to browse for the ISO image that you want to start from.
    Unselect Automatically detected from the installation media / source.
    Specify Debian 10 in the field Choose the operating system you are installing.
In step 3, allocate at least 2048 MB of RAM.
In step 4, unselect Enable storage for this virtual machine.
In step 5:
    Type a name of your choice for the new virtual machine.
    Click Finish to start the virtual machine.
\end{verbatim}


