\chapter{systemctl}



\section{systemctl}

\section{timers}
\subsection{Introduction}
Timers are systemd unit files whose name ends in \verb+.timer+ that control
\verb+.service+ files or events. Timers can be used as an alternative to cron.
Timers have built-in support for calendar time events, monotonic time events,
and can be run asynchronously. 

Timers are defined as one of two types:
\begin{itemize}
    \item Realtime timers (a.k.a. wallclock timers) activate on a calendar
        event, the same way that cronjobs do. The option \verb+OnCalendar=+ is used to define them.
    \item Monotonic timers activate after a time span relative to a varying
        starting point. They stop if the computer is temporarily suspended or
        shut down. There are number of different monotonic timers but all have
        the form: \verb+OnTypeSec=+. Common monotonic timers include
        \verb+OnBootSec+ and \verb+OnUnitActiveSec+.
\end{itemize}

For each \verb+.timer+ file, a matching \verb+.service+ file exists. The
\verb+.timer+ file activates and controls the \verb+.service+ file. The
\verb+.service+ does not require an \verb+[Install]+ section as it is the timer
units that are enabled. If necessary, it is possible to control a
differently-named unit using the \verb+Unit=+ option in the timer's
\verb+[Timer]+ section. 

\subsection{Management}

To use a timer unit \verb+enable+ and \verb+start+ it like any other unit
(remember to add the \verb+.timer+ suffix). 

To list timers use \verb+systemctl list-timers --all+

If a timer gets out of sync, it may help to delete its \verb+stamp-*+ file in
\verb+/var/lib/systemd/timers+ (or \verb+~/.local/share/systemd/+ in case of
user timers). These are zero length files which mark the last time each timer
was run. If deleted, they will be reconstructed on the next start of their
timer.

\verb+systemctl status TIMER_NAME+


\subsection{Example}

\subsubsection{Monotonic timer}

A timer which will start 15 minutes after boot and again every week while the
system is running.
\verb+/etc/systemd/system/foo.timer+:
\begin{verbatim}
[Unit]
Description=Run foo weekly and on boot

[Timer]
OnBootSec=15min
OnUnitActiveSec=1w

[Install]
WantedBy=timers.target
\end{verbatim}

\subsubsection{Realtime timer}

A timer which starts once a week (at 12:00am on Monday). When activated, it
triggers the service immediately if it missed the last start time (option
Persistent=true), for example due to the system being powered off.

\verb+/etc/systemd/system/foo.timer+

\begin{verbatim}
[Unit]
Description=Run foo weekly

[Timer]
OnCalendar=weekly
Persistent=true

[Install]
WantedBy=timers.target
\end{verbatim}


When more specific dates and times are required, OnCalendar events uses the
following format:

\begin{verbatim}
# DayOfWeek Year-Month-Day Hour:Minute:Second
\end{verbatim}
An \verb+*+ may be used to specify any value and \verb+,+ may be used to list
possible values. Two values separated by \verb+..+ indicate a contiguous range. 

\begin{verbatim}
# DayOfWeek Year-Month-Day Hour:Minute:Second

# run the first four days of each month at 12:00 PM,only if that day is a Monday or a Tuesday.
OnCalendar=Mon,Tue *-*-01..04 12:00:00

# run every day at 4am
OnCalendar=*-*-* 4:00:00
\end{verbatim}


\subsection{journalctl}
output \verb+-o+ after chose between short-full, verbose, json or cat

periode of time
\verb+-S+ (since) or \verb+-U+ (until)
journalctl -S "2020-91-12 07:00:00"
